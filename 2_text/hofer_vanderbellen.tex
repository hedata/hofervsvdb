% This is LLNCS.DEM the demonstration file of
% the LaTeX macro package from Springer-Verlag
% for Lecture Notes in Computer Science,
% version 2.4 for LaTeX2e as of 16. April 2010
%
\documentclass{llncs}
%
\usepackage{makeidx}  % allows for indexgeneration
%
\begin{document}
%
\frontmatter          % for the preliminaries
\mainmatter              % start of the contributions
%
\title{Broadcasting or Engaging? The Use of the Social Media Tool Twitter by Austrian
	Politicians during the Presidential Election 2016}
%
%\titlerunning{Hamiltonian Mechanics}  % abbreviated title (for running head)
%                                     also used for the TOC unless
%                                     \toctitle is used
%
%\author{Erich Heil}
%
%\authorrunning{Ivar Ekeland et al.} % abbreviated author list (for running head)
%
%%%% list of authors for the TOC (use if author list has to be modified)
%\tocauthor{Ivar Ekeland, Roger Temam, Jeffrey Dean, David Grove,
%Craig Chambers, Kim B. Bruce, and Elisa Bertino}
%
%\institute{Technical University, Vienna,\\
%\email{erichheil@gmail.com}
%}

\maketitle              % typeset the title of the contribution

%\begin{abstract}
%
%\keywords{Social Media, Election, Political Communication, Twitter, Content Analysis}
%\end{abstract}
%
\section{Introduction}
%
The presidential campaign is a period preceding elections where political candidates do an organized effort to persuade citizens to give them their vote. During this period politicians use different techniques to get their messages to potential voters, such as mass meetings, rallies or nowadays social media tools such as Twitter and Facebook. Understanding and exploiting the available channels better than your opponent can make the difference. 

Since Obama's heavy usage of social media tools during the 2008 U.S presidential election campaign\cite{aaker2010obama}, the importance of social media for elections is evident. Studies found that politicians with higher social media engagement received relatively more votes during the 2010 national election in the Netherlands\cite{effing2011social} and that during the Swedish general election in 2010 spikes in Twitter activity were linked to televised debates or media coverage of off-line events such as political rallies\cite{larsson2012studying}.

In the case of Austrian elections, empirical research regarding the use of social media Tools especially Twitter is, to the best of our knowledge, missing. This leaves the research community with a lack of knowledge. To fill this gap and foster our understanding of social media tool use of politicians in Austria during elections, we gathered and systematically analyzed all "tweets" of the two candidates for the 2016 presidential election, Alexander Van der Bellen and Norber Hofer, during their campaign. 

\textbf{Research Questions}: To what extent did Hofer and Van der Bellen use Twitter during their campaign, and how did the usage change? What were the main classes of Twitter engagement used by the candidates - Are they simply broadcasting their messages or are they engaging and interacting with the public?

\textbf{Approach}: To answer our first research we present and apply descriptive analyses of the data. The second research question is tackled through a content analysis of random sample tweets. The sample size equates to a 95\% confidence interval and a +-5\% margin of error. The remaining tweets are then independently grouped into five different classes. The chosen classes are similar to \cite{larsson2012studying} and \cite{miller2015studying} to provide comparability. 

\textbf{Contribution}: The contributions of our work are twofold: First we present and discuss a descriptive analysis covering the amount and topology of twitter usage of the presidential candidates during the 2016 Austrian election. Second we perform the first content analysis and classification of all the presidential candidates tweets during the 2016 election.
Our main results are: 
\begin{itemize} 
	\item Alexander Van der Bellen was more active and could generate more reactions than his opponent on twitter.
	\item Both candidates mainly used twitter to broadcast their message and inform about upcoming events such as television shows or rallies.  
\end{itemize}

Our results are in line with a study performed for the 2010 UK general election\cite{graham2013between}. In future research we will investigate if there is a correlation between social media engagement and election results.

The paper is structured as follows: After related work we describe and argue our methodology in detail. We then present the descriptive analysis and the content analysis. Finally, we discuss the empirical results and give an outlook on future work.

%
% ---- Bibliography ----
%
\begin{thebibliography}{5}
%
\bibitem{aaker2010obama}
Jennifer Aaker and Victoria Chang.
\newblock Obama and the power of social media and technology.
\newblock {\em The European Business Review}, pages 17--21, 2010.

\bibitem{effing2011social}
Robin Effing, Jos van Hillegersberg, and Theo Huibers.
\newblock Social media and political participation: are facebook, twitter and
youtube democratizing our political systems?
\newblock In {\em International Conference on Electronic Participation}, pages
25--35. Springer, 2011.

\bibitem{graham2013between}
Todd Graham, Marcel Broersma, Karin Hazelhoff, and Guido van't Haar.
\newblock Between broadcasting political messages and interacting with voters:
The use of twitter during the 2010 uk general election campaign.
\newblock {\em Information, Communication \& Society}, 16(5):692--716, 2013.

\bibitem{larsson2012studying}
Anders~Olof Larsson and Hallvard Moe.
\newblock Studying political microblogging: Twitter users in the 2010 swedish
election campaign.
\newblock {\em New Media \& Society}, 14(5):729--747, 2012.

\bibitem{miller2015studying}
Noah~W Miller and Rosa~S Ko.
\newblock Studying political microblogging: Parliamentary candidates on twitter
during the february 2012 election in kuwait.
\newblock {\em International Journal of Communication}, 9:21, 2015.

\bibitem{tumasjan2010predicting}
Andranik Tumasjan, Timm~Oliver Sprenger, Philipp~G Sandner, and Isabell~M
Welpe.
\newblock Predicting elections with twitter: What 140 characters reveal about
political sentiment.
\newblock {\em ICWSM}, 10:178--185, 2010.



\end{thebibliography}

\clearpage
\addtocmark[2]{Author Index} % additional numbered TOC entry
\renewcommand{\indexname}{Author Index}
\printindex
\clearpage
\end{document}
